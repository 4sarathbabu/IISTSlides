\documentclass{beamer}
\usepackage{graphicx}
\usepackage{hyperref}
\usepackage[ruled, linesnumbered]{algorithm2e}


\setbeamertemplate{bibliography item}[text]
\usepackage{biblatex}
%\usepackage[style=numeric, backend=bibtex]{biblatex}

\usepackage{sansmathaccent}
\pdfmapfile{+sansmathaccent.map}

\renewcommand{\bibfont}{\tiny}

%% Add your bibliography, i.e., bibtex file here
\addbibresource{pres.bib}

%% Using the iist theme
\usetheme{iist}

%% Assigning the date. You can use custom date style as 
%% \date{01 January, 2015}
\date{\today}

%% Add title, subtitle, authors
\title[$\quad\quad$ Short title]{Title}
\subtitle{}
\author[Author-1 \& Author-2]{Author 1 and Author 2}

%% Advisor/research supervisor is optional. If you don't want just 
%% comment the line.
\advisor{Research advisor}{Name}

%% You can add a figure at the bottom of the title page. It will be 
%% nice if you can add a logo of the conference that you are presenting 
%% in your slides. If you want logo, just uncomment this line.
%% \bottomfigure{\hspace*{9cm}\includegraphics[scale=0.1]{figures/conf.png}}


%% Assigning institute name. If you want department, just add 
%% before the Institute name.
\institute[IIST]{Indian Institute of Space Science and Technology \\
Thiruvananthapuram, Kerala, India 695547 \\}

\begin{document}

%% Creating title slide
\begin{frame}[plain]
	\titlepage
\end{frame}

%% Creating slide with table of contents
\begin{frame}
	\frametitle{Outline}
	\tableofcontents
\end{frame}

%% Here onwards you can add your contents
\section{Listing}
\subsection{Itemization}
\begin{frame}
	\frametitle{Itemize}
		\begin{itemize}
 			\item Item 1
 			\item Item 2
 			\begin{itemize}
				\item Subitem 2.1
				\begin{itemize}
					\item Subsubitem 2.1.1
					\item Subsubitem 2.1.2
				\end{itemize}
				\item Subitem 2.2
 			\end{itemize}
 			\item Item 3
 		\end{itemize} 		
\end{frame}


\subsection{Enumeration}
\begin{frame}
	\frametitle{Enumerate}
	\begin{enumerate}					
		\item Item 1
		\item Item 2
		\item Item 3
	\end{enumerate}	
\end{frame}


\subsection{Description}
\begin{frame}
	\frametitle{Description}
		\begin{description}
			\item [One] Item 1
			\item [Two] Item 2
		\end{description}
\end{frame}	

\section{Blocks}
\subsection{Different blocks}
\begin{frame}
	\frametitle{Blocks}
		\begin{block}{Block}
			This is a sample block
		\end{block}
\end{frame}
\begin{frame}
		\frametitle{Example block}
		\begin{exampleblock}{Example}
			This is a sample example block
		\end{exampleblock}
\end{frame}
\begin{frame}
		\frametitle{Alert block}
		\begin{alertblock}{Alert}
			This is a sample alert block
		\end{alertblock}
\end{frame}	

\section{Figures \& Columns}
\subsection{Figures}
\begin{frame}
	\frametitle{Including figure}
	\begin{figure}
		\centering
		\includegraphics[scale=0.1]{figures/latexlogo.png}
		\caption{\LaTeX-logo}
	\end{figure}
\end{frame}

\subsection{Columns}
\begin{frame}
	\frametitle{Multiple columns}
	\begin{columns}
		\begin{column}{0.5\paperwidth}
		 \begin{figure}
			\centering
			\includegraphics[scale=0.1]{figures/latexlogo.png}
			\caption{\LaTeX-logo}
			\end{figure}
		\end{column}
		\begin{column}{0.5\paperwidth}
		 \begin{itemize}
			\item Item 1
			\item Item 2
			\item Item 3
		 \end{itemize}
		\end{column}	
	\end{columns}
\end{frame}

\section{Algorithms}
\subsection{Sample algorithm}
\begin{frame}
	\frametitle{Algorithm}
	\begin{algorithm}[H]
		%\SetAlgoVlined
		\caption{Sum of $N$ numbers}
		$S = 0$\\
		\For{$i=1:N$}
		{
			$S = S + i$
		}
		Ouput $S$
	\end{algorithm}
\end{frame}

\section{Bibliography}
\subsection{Footnote}
\begin{frame}
	\frametitle{Citation} 
	\begin{block}{}
	``GNU, which stands for Gnu's Not Unix, is 
	the name for the complete Unix-compatible software system which 
	I am writing so that I can give it away free to everyone who can 
	use it\footnotemark."
	\end{block}
	\footnotetext{\fullcite{stallman1985gnu}}
	EMACS is a real-time display editor which can be extended by the 
	user while it is running\footfullcite{stallman1981emacs}.
\end{frame}

%% Since the theme displays references as footnotes, it is not 
%% recommended to make it redundant again in the last slide. Still, if 
%% you want references as a separate slide, uncomment the following 
%% frame.

%\subsection{References}
%\begin{frame}
	%\frametitle{References}
	%\nocite{*}
	%\printbibliography
%\end{frame}

\customframe{Questions?}{abc@mail.com}
\customframe{Thank you.}{}


\end{document}
